\documentclass[10pt,a4paper,german]{scrartcl}
\usepackage[utf8]{inputenc}

\usepackage{acl2005}

\usepackage[german]{babel}
\usepackage{amsmath}
\usepackage{amsfonts}
\usepackage{amssymb}
\usepackage{graphicx}
\usepackage{lmodern}
%\usepackage{kpfonts}
%\usepackage{fourier}
%\usepackage[left=2cm,right=2cm,top=2cm,bottom=2cm]{geometry}

%\usepackage{times}
%\usepackage{latexsym}

\setlength\titlebox{2.5cm}    % Expanding the titlebox

% Solution of the Schrödinger equation by a spectral method ☆
% M.D Feit, J.A Fleck Jr., A Steiger
% http://dx.doi.org/10.1016/0021-9991(82)90091-2

% Time-dependent propagation of high energy laser beams through the atmosphere 
% J. A. Fleck, J. R. Morris and M. D. Feit
% http://dx.doi.org/10.1007/BF00896333

\title{Energiespektrum aus Autokorrelationsfunktion}
\author{Pavel Sterin}

\begin{document}
	\maketitle
	\section{Split-Operator Methode}
		Die Split-Operator Methode ist ein Verfahren zur numerischen Approximation
		von Lösungen der Schrödingergleichung in kartesichen Koordinaten. Zur Vereinfachung
		der Berechnungen setze ich verwende ich die Konvention $\hbar = 1$, $m=\frac{1}{2}$.
		Damit erhält man eine einfache Form, der linearen DGL:

		\begin{align*}
				&\left(-\frac{\partial^2}{\partial x^2} + V(x,t)\right)\psi(x,t) 
					= i \frac{\partial}{\partial t}\psi(x,t) \\
				&= \left(T + V\right)\psi(x,t)
					= i \frac{\partial}{\partial t}\psi(x,t)
		\end{align*}
		
		Mit dem üblichen Ansatz für die zeitliche Evaluation von $ \psi(x,t)=U(t)\psi(x) $
		(mit fixierter Wellenfunktion $\psi(x)$ und unitärem Zeit-Propagator $U(t)$) erhält man
		die äquivalente Schrödingergleichung des Propagators:
		\begin{equation*}
			\left(T + V(t)\right) U(t) = i  \frac{\partial}{\partial t}U(t)
		\end{equation*}
		Die formale Lösung dieser Gleichung ist bekannterweise die Dyson Reihe, oder etwas
		weniger allgemein für verschiedene Zeiten mit sich selbst kommutierenden Hammiltonian
		$H(t)=T+V(t)$, $[H(t),H(t')]=0$ ist:
		\begin{equation*}
			U(t)=\exp\left(\int_{0}^{t}{-i (T+V(t')) \mathrm{d} t'}\right)		
		\end{equation*}
		Falls sich das Potential für kleine Zetdiffrenzen $\tau$ nur geringfügig ändert oder
		sogar stets konstant ist, sodass $[H(t),H(t+\tau)]\approx 0$ gilt, geht diese Lösung
		in das Operator-Exponential über:
		\begin{equation} \label{eq:expform}
			U(t,\tau)=\exp\left(-i \tau (T+V(t))\right)
		\end{equation}
		
  \section{Autokorrelationsfunktion}
  \section{Diskrete Fourier-Transformation}
  \section{Eigenarten einiger Spektra}
\end{document}