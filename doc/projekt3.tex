\documentclass[10pt,a4paper,german]{scrartcl}
\usepackage[utf8]{inputenc}

\usepackage{acl2005}

\usepackage[german]{babel}
\usepackage{amsmath}
\usepackage{amsfonts}
\usepackage{amsthm}
\usepackage{amssymb}
\usepackage{graphicx}
\usepackage{lmodern}

\usepackage{trsym}
\usepackage{dsfont}
\usepackage{pifont}

%\usepackage{kpfonts}
%\usepackage{fourier}
%\usepackage[left=2cm,right=2cm,top=2cm,bottom=2cm]{geometry}

%\usepackage{times}
%\usepackage{latexsym}

\setlength\titlebox{2.5cm}    % Expanding the titlebox

% Solution of the Schrödinger equation by a spectral method ☆
% M.D Feit, J.A Fleck Jr., A Steiger
% http://dx.doi.org/10.1016/0021-9991(82)90091-2

% Time-dependent propagation of high energy laser beams through the atmosphere 
% J. A. Fleck, J. R. Morris and M. D. Feit
% http://dx.doi.org/10.1007/BF00896333

% Continuous and Discrete Time Signals and Systems [Hardcover]
% Mrinal Mandal, Amir Asif

\title{Energiespektrum aus Autokorrelationsfunktion}
\author{Pavel Sterin}

\begin{document}
	\maketitle
	\section{Split-Operator Methode}
		Die Split-Operator Methode ist ein Verfahren zur numerischen Approximation
		von Lösungen der Schrödingergleichung in kartesichen Koordinaten. Zur Vereinfachung
		der Berechnungen setze ich verwende ich die Konvention $\hbar = 1$, $m=\frac{1}{2}$.
		Damit erhält man eine einfache Form, der linearen DGL:

		\begin{align*}
				&\left(-\frac{\partial^2}{\partial x^2} + V(x,t)\right)\psi(x,t) 
					= i \frac{\partial}{\partial t}\psi(x,t) \\
				&= \left(T + V\right)\psi(x,t)
					= i \frac{\partial}{\partial t}\psi(x,t)
		\end{align*}
		
		Mit dem üblichen Ansatz für die zeitliche Evaluation von $ \psi(x,t)=U(t)\psi(x) $
		(mit fixierter Wellenfunktion $\psi(x)$ und unitärem Zeit-Propagator $U(t)$) erhält man
		die äquivalente Schrödingergleichung des Propagators:
		\begin{equation*}
			\left(T + V(t)\right) U(t) = i  \frac{\partial}{\partial t}U(t)
		\end{equation*}
		Die formale Lösung dieser Gleichung ist bekannterweise die Dyson Reihe, oder etwas
		weniger allgemein für verschiedene Zeiten mit sich selbst kommutierenden Hammiltonian
		$H(t)=T+V(t)$, $[H(t),H(t')]=0$ ist:
		\begin{equation*}
			U(t)=\exp\left(\int_{0}^{t}{-i (T+V(t')) \mathrm{d} t'}\right)		
		\end{equation*}
		Falls sich das Potential für kleine Zetdiffrenzen $\tau$ nur geringfügig ändert oder
		sogar stets konstant ist, sodass $[H(t),H(t+\tau)]\approx 0$ gilt, geht diese Lösung
		in das Operator-Exponential über:
		\begin{equation*}
			U(t,\tau)=\exp\left(-i \tau (T+V(t))\right)
		\end{equation*}
		Im weiteren beziehe ich mich stets auf eine einzige Iteration und lasse darum die
		zetliche Abhängikeit von $V$ fallen:
		\begin{equation}\label{eq:exp}
			U(\tau)=\mathrm{e}^{-i \tau (T+V)}
		\end{equation}
		Da $T$ nur im Impulsraum und $V$ nur im Ortsraum wirken, wäre es günstig sie nur dort
		auszuwerten, weil sie dann als Multiplikationsopretoren diagonal wären und
		das Exponential sich leicht auswerten ließe. Da jedoch für die meisten interessanten 
		Probleme $T$ und $V$ nicht Kommutieren, $[T,V]\neq 0$, muss man hier geschickt
		Approxomieren. 
		
		Für jeden Operator gilt zunächst:
		\begin{equation*}
			\mathrm{e}^{- i \tau X}
				= \sum_{n=0}^{\infty} \frac{(i \tau X)^n}{n!}
				= 1 - i \tau X - \tau^2 \frac{X^2}{2} + O(\tau^3)
		\end{equation*}
		
		Mit den Bezeichnung	$U_X(\tau)=\mathrm{e}^{-i \tau X}$ ergibt das dann:
		\begin{align*}
			U_{X+Y}(\tau) &= 1 - i \tau (X+Y) -\tau^2 \frac{(X+Y)^2}{2} + O(\tau^3)\\
							&= 1 - i \tau X - i \tau Y\\ 
							&\quad-\frac{\tau^2}{2} \left(	X^2 + X Y + Y X + Y^2 \right) + O(\tau^3)\\
		\end{align*}
		\begin{equation*}
			U_X(\tau) = 1 - i \tau X -\tau^2 \frac{X^2}{2} + O(\tau^3)
		\end{equation*}
		\begin{equation*}
			U_Y(\tau) = 1 - i \tau Y -\tau^2 \frac{Y^2}{2} + O(\tau^3)
		\end{equation*}
		\begin{multline*}
			U_X(\tau) U_Y(\tau) = 
				1
				- i \tau Y - i \tau X\\
				- \tau^2 X Y 
				- \tau^2 \frac{Y^2}{2}
				- \tau^2 \frac{X^2}{2}
				+ O(\tau^3)
		\end{multline*}
		Damit erhält man den Ausdruck:
		\begin{equation}\label{eq:expand}
			U_{X+Y}(\tau) = U_X(\tau)U_Y(\tau) 
				+ \frac{\tau^2}{2} \left[X,Y\right]	+ O(\tau^3)\\
		\end{equation}
		Mit $X=\frac{V}{2}$, $Y=T+\frac{V}{2}$ folgt:
		\begin{multline}
			U_{\frac{V}{2}+T+\frac{V}{2}}(\tau)
				= U_{\frac{V}{2}}(\tau)U_{T+\frac{V}{2}}(\tau) \\
					+ \left[\frac{V}{2},T+\frac{V}{2}\right]	+ O(\tau^3)\\
				= U_{\frac{V}{2}}(\tau)U_{T+\frac{V}{2}}(\tau) 
					+ \left[\frac{V}{2},T\right]	+ O(\tau^3)\\
				= U_{\frac{V}{2}}(\tau)
						\left(
							U_T(\tau)U_{\frac{V}{2}}(\tau)
								+ \frac{\tau^2}{2} \left[T,\frac{V}{2}\right]
						\right)\\
					+ \left[\frac{V}{2},T\right]	+ O(\tau^3)\\
				= U_{\frac{V}{2}}(\tau) U_T(\tau) U_{\frac{V}{2}}(\tau) + O(\tau^3)
		\end{multline}
		Durch die symmterische Aufteilung des Potentials verringert sich also
		der Fehler auf die 3. Ordnung.
		
		In dieser Darstellung ist jetzt einfach die Wirkung von
		$U \approx \tilde{U}=U_{\frac{V}{2}}(\tau) U_T(\tau) U_{\frac{V}{2}}(\tau)$ zu berechnen,
		denn:
		\begin{align}
		\label{it:hV}
			U_{\frac{V}{2}}(\tau) \psi(x) &= \mathrm{e}^{-i \frac{\tau}{2} V(x)} \psi(x)\\
		\label{it:T}
			U_T(\tau) \psi(x) &= \mathcal{F}^{-1} \mathrm{e}^{-i \tau k^2} \mathcal{F} \psi(x)
		\end{align}
		wobei $\mathcal{F}$ die Fouriertransformation ist.
		
		Diese beiden Gleichungen definieren die Iteration der Split-Operator-Methode
		für $n$ Schtitte bzw. Zeit $t=n \tau$
		
		\begin{description}
			\item[Start]
				\begin{itemize}
					\item Propagation von $\psi(x)$ mit $U_{\frac{V}{2}}(\tau)$
					\item Fouriertransformation zu $\psi(k)$
					\item Propagation von $\psi(k)$ mit $U_T(\tau)$
				\end{itemize}
			\item[Iteration]
				\begin{itemize}
					\item inverse Fouriertransformation zu $\psi(x)$
					\item Propagation von $\psi(x)$ mit $U_{\frac{V}{2}}(\tau)$
					\item Propagation von $\psi(x)$ mit $U_{\frac{V}{2}}(\tau)$
					\item Fouriertransformation zu $\psi(k)$
					\item Propagation von $\psi(k)$ mit $U_T(\tau)$
				\end{itemize}
			\item[Ende]
				\begin{itemize}
					\item inverse Fouriertransformation zu $\psi(x)$
					\item Propagation von $\psi(x)$ mit $U_{\frac{V}{2}}(\tau)$
				\end{itemize}
		\end{description}
		wobei die Iteration $(n-1)$ Mal ausgeführt wird. Die zwei nacheinander folgenden
		Propagationen mit $U_{\frac{V}{2}}(\tau)$ werden in der Implementierung natürlich
		zu $U_V(\tau)$ zusammengefasst.
		
		\textbf{Bemerkung:} Es hindert einen niemend daran die Rollen von T und V zu
		vertauschen	und statt $\tilde{U}$ den Operator 
		$U_{\frac{T}{2}}(\tau) U_V(\tau) U_{\frac{T}{2}}(\tau)$ zu verwenden. Der einzige
		Unterschied ist eine zurätliche Fouriertransformation am Anfang und eine 
		inverse Fouriertransformation am Ende des Algorithmus, die mir angesichts der 
		eher kurzen Iterationen für dieses Projekt unangebracht erschienen.

		
  \section{Autokorrelationsfunktion}
		Ein zeitunabhängiger Hamiltonian $H$ eines Systems besitzt eine Basis in der er als 
		Multiplikationsoperator wirkt. In dieser Basis besitzt $H$ eine Spektralzerlegung
		der Form 
		\begin{equation*}
			H = \int_{\sigma(H)}{\epsilon \mathrm{d}\mu(\epsilon)}
		\end{equation*}
		mit Spektralmaß $\mu(\epsilon)$ und Spektrum $\sigma(H)$.
		Da $H$ nun ein Multiplikationsoperator ist, lässt sich der
		Zeit-Propagator formal einfach auswerten zu:
		\begin{equation*}
			U(t)=\exp\left(
				-i t \int_{\sigma(H)}{\epsilon \mathrm{d}\mu(\epsilon)}
			\right)
			=\int_{\sigma(H)}{\mathrm{e}^{-i t \epsilon} \mathrm{d}\mu(\epsilon)}
		\end{equation*}
		Für die Autokorrelationsfunktion $c(t)$
		gilt dann dementsprechend:
		\begin{multline}
			\label{eq:corr}
			c(t) = \langle \psi,U(t) \psi\rangle
					 = \int_{x} \psi^{*} U(t) \psi \\
					 = \int_{x} \psi^{*}
					 	  \int_{\sigma(H)}{\mathrm{e}^{-i t \epsilon}
					 	 	   \mathrm{d}\mu(\epsilon)} \psi \\
					 = \int_{\sigma(H)}{
					 			\mathrm{e}^{-i t \epsilon}
					 			\int_{x} \psi^{*}
					 	 	   \mathrm{d}\mu(\epsilon)} \psi \\
					 = \int_{\sigma(H)}{
					 			\mathrm{e}^{-i t \epsilon}
					 			\mathrm{d} \tilde{\mu}(\epsilon)}
		\end{multline}
		
		Die inverse Fouriertransformation von $c(t)$ gibt ein moduliertes Spektrum
		der Energieeigenwerte von H.
		
		\begin{multline}
			\label{eq:corrft}
			c(\epsilon) = \mathcal{F}[c](\epsilon) 
			      = \frac{1}{\sqrt{2 \pi}}
			     		\int_{\mathbb{R}} c(t) \mathrm{e}^{it \epsilon} \mathrm{d}t\\
			     = \frac{1}{\sqrt{2 \pi}}
		     			\int_{\sigma(H)}
				     		\int_{\mathbb{R}} \mathrm{e}^{i t (\epsilon -\epsilon')}	\mathrm{d}t
					 		\mathrm{d} \tilde{\mu}(\epsilon')\\
			     = \sqrt{2 \pi}	
			     		\int_{\sigma(H)} \delta(\epsilon -\epsilon')
					 		\mathrm{d} \tilde{\mu}(\epsilon')
		\end{multline}
		
		Um die Bedeutung von $c(\epsilon)$ besser zu verstehen, betrachte man einen
		etwas weniger allgemeinen Fall eines Systems mit isolierten Eigenwerten
		($H$ kompakt und normal da selbtstadjungiert). In einer Basis in der 
		dieser Hamiltonian diagonal ist hat jeder Zustand zie Zerlegung
		$\psi=\sum_{j}{a_j e_j}$ mit Eigenfunktionen $e_j$ zu Eigenenergie $\epsilon_j$.
		Die Autokorrelation ist nach \eqref{eq:corr}
		$c(t)= \sum_{j} |a_j|^2 \mathrm{e}^{-i t \epsilon_j}$ und das modulierte Spektrum
		ist damit nach \eqref{eq:corrft}:
		\begin{equation}
			c(\epsilon)=\sqrt{2 \pi}
				 \sum_{j} |a_j|^2 \mathrm{e}^{-i t \epsilon_j} \delta(\epsilon-\epsilon')
		\end{equation}
		Die $\delta$-Distribution in der letzten Formel kommt natürlich nur dann zustande,
		wen über ganz $\mathbb{R}$ integriert wird. Für alle praktischen Zwecke steht
		aber nur ein begrenztes Intervall $I=[0:T]$ zur Verfügung. Statt des echten Spektrums
		bekommt man nach nur die Faltung von $c(\epsilon)$ mit $\mathcal{F}[w](\epsilon)$,
		wobei $w(t)$ die Rechteck-Fensterfunktion mit Träger in I ist. Nach Faltungstheorem
		erhält man:
		\begin{multline}
			\mathcal{F}[c w](\epsilon)
				= \int_{\mathbb{R}} c(t) w(t) \mathrm{e}^{i t \epsilon} \mathrm{d}t \\
				= \int_{\mathbb{R}}
					 	\int_{\mathbb{R}}
					 		 c(t) w(t') \delta(t-t') \frac{2 \pi}{(\sqrt{2 \pi})^2}
					 		 	 \mathrm{e}^{i t \epsilon} 
						\mathrm{d}t
					\mathrm{d}t'\\
				= \frac{1}{(\sqrt{2 \pi})^2} \int_{\mathbb{R}}
					 	\int_{\mathbb{R}}
					 		 c(t) w(t')
					 		   \int_{\sigma(H)} \mathrm{e}^{i (t - t')\epsilon'} \mathrm{d}\mu(\epsilon')
					 		 	 \mathrm{e}^{i t \epsilon} 
						\mathrm{d}t
					\mathrm{d}t'\\
				= 
 		  	\int_{\sigma(H)} 
					\left(\int_{\mathbb{R}}
						 \frac{c(t)}{\sqrt{2 \pi}} \mathrm{e}^{i t\epsilon'} \mathrm{d}t\right)
					\left(\int_{\mathbb{R}} 
						\frac{w(t')}{\sqrt{2 \pi}}  \mathrm{e}^{i t' (\epsilon-\epsilon')}
							\mathrm{d}t'\right)
				\mathrm{d}\mu(\epsilon')\\
				= \int_{\sigma(H)} 
	 		  		\mathcal{F}[c](\epsilon') \mathcal{F}[w](\epsilon-\epsilon')
					\mathrm{d}\mu(\epsilon')\\
				= (\mathcal{F}[c] * \mathcal{F}[w])(\epsilon)
		\end{multline}

		Für $w(t)=\Theta(t)\Theta(T-t)$ erhält man
		\begin{equation*}
			W(\epsilon) =\mathcal{F}[w](\epsilon) = 
				\frac{1}{\sqrt{2\pi}}\int_{0}^{T}	\mathrm{e}^{i t\epsilon} \mathrm{d}t
			= -i \frac{\mathrm{e}^{i T E}-1}{\sqrt{2\pi} E}
		\end{equation*}
		\begin{equation*}
			|W(\epsilon)| = \sqrt{\frac{1-\cos(T \epsilon)}{\pi \epsilon^2}}
			= \frac{|\sin(\frac{T \epsilon}{2})|}{\pi |\epsilon|}
		\end{equation*}
		Das Rechteck-Fenster verschmiert also nebeneinander  liegende Peaks und verringert
		die Auflösung des Verfahrens. Man kann das umgehen, indem man eine andere
		Fenster-Funktion verwendet. Für dieses Projekt wurde die übliche
		von-Hann Funktion 
		\begin{equation}
		\label{eq:hann}
			w(t) = \left\{\begin{array}{cl}
				\frac{1-\cos(\frac{2\pi t}{T})}{2}, & t \in I=[0:T]\\
				0,                        & \mbox{sonst} 
				\end{array}\right. 
		\end{equation}
		verwendet.
  
  \section{Diskrete Fouriertransformation}
		Für eine Simulation, die in endlicher Zeit laufen soll, kann man keine
		direkte Fouriertransformation benutzen, denn die symbolische Auswertung
		der Integrale ist leider nur für sehr wenige Probleme mit kleinen Gittern
		sinnvoll implementierbar. Deswegen diskretisiert man bei Simulationen
		normalerweise den Ort (und Zeit) und geht von kontinuierlichen zu diskreten
		Fouriertransformationen über.
		
		Sei $x \in \mathbb{C}^n$ ein Vektor dann ist die DFT von $x$ definiert
		als
		\begin{equation}
			\hat{x}_\alpha
			= \tilde{\mathcal{F}}[\hat{x}]_\alpha 
			= \sum_{l=0}^{n-1} \mathrm{e}^{-2 \pi \frac{\alpha }{n} l} x_l
		\end{equation}

%		$\longrightarrow$
  \section{Eigenarten einiger Spektra}
\end{document}